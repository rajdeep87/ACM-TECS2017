\textbf{Step 4: Mining Scenario/Transaction Pairs Using Symbolic Execution}
Given a co-verification model, $\mathcal{M}_{seq}$, a scenario, 
$\mathcal{S}$, and a property expressed as $\mathit{assert}(c)$ 
(where $c$ is a condition stated in terms of variables in $\mathcal{M}_{seq}$)
as input, \verifox performs path-based exploration of $\mathcal{M}_{seq}$ 
to automatically infer {\em scencario/transaction} pairs and check their validity 
using backend solvers. If the condition $c$ does not hold, 
then $\mathcal{M}_{seq}$ is said to have violated the property. 
\rmcmt{mark algo with FT and HT scenario transaction pair}

A typical path-based symbolic execution engine might explore 
a path until it come to an $\mathit{assert}(c)$ statement.
This whole path can then be posed as a query to a SAT solver to see if the 
assertion is violated at that point. If the path is infeasible, the assertion
holds trivially. If a large number of paths are infeasible, symbolic 
execution may waste time exploring them. \verifox employs 
an eager infeasibility check to prune infeasible paths, 
as well as incremental encoding that makes
it easier for the underlying SAT solver. 
\begin{algorithm2e}[t]
\DontPrintSemicolon
\SetKwFunction{symex}{Symex}
%\SetKwFunction{assert}{assert}
\SetKwFunction{put}{put}
\SetKwFunction{get}{get}
\SetKwFunction{print}{print}
\SetKwFunction{return}{return}
%\SetKwFunction{assume}{assume}
%\SetKwFunction{isf}{isFeasible}
\SetKwData{safe}{Safe}
\SetKwData{unsafe}{Unsafe}
\SetKwInOut{Input}{input}
\SetKwInOut{Output}{output}
\SetKw{KwNot}{not}
\begin{small}
  \Input{Co-verification Model $\mathcal{M}_{seq}$ with properties specified with $assert(c)$, scenario specified with $assume(c)$}
\Output{The status (\safe or \unsafe) and a counterexample if \unsafe}
\tcc{The initial state}
$S_0 \leftarrow I(x)$ \nllabel{algline:init}\;
$stmt \leftarrow $ first statement \;
$worklist.put(\tuple{S_0,stmt})$ \;
\While{\KwNot $worklist.empty()$}
{
	$\tuple{S,stmt} \leftarrow worklist.get()$ \;
	\uIf{$stmt$ is an $assume(c)$ \nllabel{algline:assume}}
           {
			$stmt \leftarrow $ statement after $assume(c)$ \;
			\lIf{$isFeasible(S\wedge c)$}{$worklist.put(\tuple{S\wedge c,stmt})$}
              
       		}
	\uElseIf{$stmt$ is a branch with condition $c$ \nllabel{algline:branch}}
	{
		$stmt_f \leftarrow$ first statement after $stmt$ if branch is not taken\;
		$stmt_t \leftarrow$ first statement after $stmt$ if branch is taken \;
		\lIf{$isFeasible(S\wedge c)$}{$worklist.put(\tuple{S\wedge c,stmt_t})$}
		\lIf{$isFeasible(S\wedge \neg c)$}{$worklist.put(\tuple{S\wedge \neg c,stmt_f})$}
		
	}
	\uElseIf{$stmt$ is an $assert(c)$ \nllabel{algline:assert}}
	{
		$stmt \leftarrow $ statement after $assert(c)$ \;
		\If{$isFeasible(S \wedge \neg c)$}{
		\print \unsafe \;
		\return Counterexample}
		\lElse{ $worklist.put(\tuple{S \wedge c,stmt})$}
	}
	\Else
	{
		$S \leftarrow symex(S,stmt)$\; \nllabel{algline:symex}
		$stmt \leftarrow $ the next statement in control flow after $stmt$ \;
		\lIf{$stmt \neq \bot$ \nllabel{algline:nullstmt}} {$worklist.put(\tuple{S,stmt})$} \;
		
	}
	\return \safe \;
}
\end{small}
\caption{Mine Scenario/Transaction Pair Using Symbolic Execution\label{Alg:verifox}}
\end{algorithm2e}

\Omit{
, in which every time a branch or an assumption
is encountered, a feasibility check is made to prune away infeasible 
paths as early as possible. }
%\vspace{-4mm}
\algref{verifox} shows the overall algorithm of \verifox. States mentioned
in the algorithm are all symbolic states, which are quantifier-free predicates characterizing
a set of program states. 
Symbolic execution starts with an initial symbolic state $I(x)$, is a quantifier-free predicate
over program variables $x$, and the first statement $\mathit{stmt}$ to be executed.  Note that we assume all program variables
have finite bit-width and thus can be represented as bit-vectors.
Every statement acts as a state transformer during the symbolic execution. $\mathit{worklist}$ 
maintains the set of symbolic states, along with the corresponding
$\mathit{stmt}$ that should execute next. Assumptions can be specified in the program 
using $\mathit{assume}(c)$ statements, where $c$ is the condition expressed in terms 
of program variables. Assumptions restrict the search to only those
states for which the condition $c$ holds at the program point where $\mathit{assume}(c)$ is encountered.
For example, suppose $(x!=0)$ characterizes the set of states that has been discovered to be
reachable so far by a verification tool. Here, $x$ is a program variable.
Upon encountering $\mathit{assume}(x>0)$, the 
set of states reachable at the point of assumption is shrunk to only those that
satisfy $(x>0)$. A user can specify assumptions to restrict
the verification to only certain regions of the program's state space.

\verifox performs a feasibility check at an $\mathit{assume}$ statement 
or a branch \algtwolines{algline:assume}{algline:branch} to ensure that
only feasible symbolic states are kept in the $\mathit{worklist}$. This ensures that the
infeasibility is detected as early as possible. 
If an assertion is violated, then a counterexample is detected and \algref{verifox}
terminates \algline{algline:assert}. In all other cases, $\mathit{symex}(S,\mathit{stmt})$ performs
one step of symbolic execution by executing $\mathit{stmt}$ from the symbolic state $S$ \algline{algline:symex}.
If no further statement remains to be executed along the path that is being explored,
then $\mathit{stmt}$ is assigned the value $\bot$. The symbolic state is put in the worklist only 
if there are further statements remaining \algline{algline:nullstmt}.

The feasibility checks shown as $\mathit{isFeasible}$ pose a query to the underlying SAT solver.
Note that \algref{verifox} does not refer to how the methods $\mathit{worklist.put}$ and $\mathit{worklist.get}$
work. In principle, one can use any search heuristic to select which symbolic state to explore 
further from the $\mathit{worklist}$. In the current version \verifox employs a depth first strategy of exploration.

Apart from the eager infeasibility check, another crucial optimization is the
use of incremental SAT solving during the symbolic execution. 
%\verifox can use incremental SAT-solving capabilities in two ways.
During the symbolic execution, only one solver instance is maintained 
while going down
a single path. Thus, when making a feasibility
check from one branch $b_1$ to another branch $b_2$ along a single path, only the program
segment from $b_1$ to $b_2$ is encoded as a constraint and added to the existing solver
instance. 
\Omit{
Internal solver states and the information that it gathers during the search
remains valid as long as all the queries that are posed to the solver in succession are
monotonically stronger. If the solver solves a formula $\phi$, posing $\phi \wedge \psi$ as 
a query to the same solver instance allows one to reuse the solver knowledge that
it has acquired in the previous query, because any assignment that falsifies $\phi$ also
falsifies $\phi \wedge \psi$. Thus the solver need not revisit the assignments that it has already
ruled out.}
This results in speeding up the process of feasibility check of the symbolic state at $b_2$ as
the feasibility check at $b_1$ was $\mathit{true}$. A new
solver instance is used to explore a different path, after the current path is detected as infeasible.

\Omit{
In the full incremental mode, only one solver instance is maintained throughout the whole
 symbolic execution. Let $\phi_{b_1b_2}$ denote the encoding of the path fragment
from $b_1$ to $b_2$. It is added in the solver as $(B_{b_1b_2} \implies \phi_{b_1b_2})$.
Then, $B_{b_1b_2}$ is added as a \textit{blocking variable}, which is also usually known as {\em solver assumption}.\footnote{The SAT community uses the term \textit{assumption variables} or \textit{assumptions}, but we will use the term blocking variable to avoid ambiguity with assumptions in the program.} to enforce constraints specified by $\phi_{b_1b_2}$. Blocking variables are treated specially inside the solvers: unlike regular variables or clauses, the blocking can be removed in subsequent queries without invalidating the solver instance.
When one wants to back-track the symbolic execution, the blocking $B_{b_1b_2}$ is removed 
and a unit clause $\neg B_{b_1b_2}$ is added to the solver, thus effectively removing $\phi_{b_1b_2}$.
}

Eager infeasibility check restrict the search to explore
only those FW/HW interations which are feasible under a
given scenario.
%Eager infeasibility check may save the engine from exploring multiple paths
%that emerge through branches encountered after the first point of infeasibility.
In our experiments, we find this optimization has a large effect on runtimes.
Though \verifox poses many queries to the SAT solver, each query is relatively simple due to
two reasons: the resultant formula encodes only a single path, 
and exploration along a path only needs to encode and solve for the path segment (along with the existing
constraints) from the last point of query.
